\documentclass[10pt]{article}

\usepackage[a4paper,total={17cm,25cm}]{geometry}
\usepackage[fleqn]{amsmath}
\usepackage{amssymb}
\usepackage{mathrsfs}
\usepackage{amsthm}
\usepackage{dsfont}
\usepackage[mathcal]{euscript}
\usepackage[colorlinks=true,citecolor=blue]{hyperref}
\usepackage{cleveref}
\usepackage{graphicx}

\newtheorem{thm}{Theorem}
\newtheorem{lem}[thm]{Lemma}
\newtheorem{prop}[thm]{Proposition}
\newtheorem{cor}[thm]{Corollary}

\theoremstyle{definition}
\newtheorem{defn}{Definition}
\newtheorem{axm}{Axiom}
\newtheorem{conj}{Conjecture}
\newtheorem{exmp}{Example}

\crefname{lem}{Lemma}{Lemmas}
\crefname{thm}{Theorem}{Theorems}

\theoremstyle{remark}
\newtheorem*{rem}{Remark}
\newtheorem*{note}{Note}

\newenvironment{prf}{\noindent\textbf{Proof.}}{\hfill$\blacksquare$}

\begin{document}

\begin{center}
{\Large \scshape Well-Ordering and Induction} \\
Written by Hosein Rahnama
\end{center}

\begin{axm}
Every nonempty subset $S$ of the natural numbers $\mathbb{N}$ has a minimum. This is called the well-ordering principle.
\end{axm}

\begin{axm}
Let $S$ be a subset of $\mathbb{N}$ such that $1 \in S$ and $n \in S$ implies that $n + 1 \in S$. Then $S$ is equal to $\mathbb{N}$. This is called the induction principle.
\end{axm}

\begin{thm}
The well-ordering principle is equivalent to the induction principle.
\end{thm}

\begin{prf}
Both proofs are by contradiction. First, let us show that the well-ordering principle implies the induction principle. Let $S$ be a subset of $\mathbb{N}$ such that $1 \in S$ and $n \in S$ implies that $n + 1 \in S$. We want to prove that $S = \mathbb{N}$. Suppose that it is not true then the complement of $S$ defined as $T = \mathbb{N} - S$ is nonempty, and by the well-ordering principle has a minimum $t_{\min} \in T$. Since $1 \notin T$ then $1 < t_{\min}$ and $t_{\min} - 1 \in \mathbb{N}$. Furthermore, $t_{\min} -1 < t_{\min}$ which implies that $t_{\min} - 1 \notin T$ since for every $t \in T$ we have $t_{\min} \leq t$.  Consequenlty, $t_{\min} - 1$ should belong to the complement of $T$ which is $\mathbb{N} - T = \mathbb{N} - (\mathbb{N} - S) = S$ so $t_{\min} - 1 \in S$. The inductive property of $S$ tell us that $(t_{\min} - 1) + 1 = t_{\min} \in S$, which is in contradiction with $t_{\min} \in T$ and we should have $S=\mathbb{N}$.

Now, let us prove that the induction principle implies the well-ordering principle. Let $S$ be a nonempty subset of $\mathbb{N}$. We want to show that it has a minimum. Suppose that this is not true then $1 \notin S$ otherwise $S$ would have a minimum since for every $s \in S$ we have $1 \leq s$. This means that $S \subset \mathbb{N} - \{ 1 \}$. Now, $2 \notin S$ otherwise $2$ would be a minimum for $S$ since for every $s \in S$, we know that $s \in \mathbb{N} - \{ 1 \}$ which implies $2 \leq s$. This leads to $S \subset \mathbb{N} - \{1, 2\}$. Let us define $T = \{ t | \, S \subset \mathbb{N} - \{ 1, 2, \dots, t\}  \}$. We observe that $1 \in T$. Furthermore, if $t \in T$ then $t + 1 \in T$. Indeed, $t \in T$ tells us that $S \subset \mathbb{N} - \{ 1, 2, \dots, t\}$. We claim that $t + 1 \notin S$ otherwise for every $s \in S$ we know that $s \in \mathbb{N} - \{1, 2, \dots, t\}$, implying that $t + 1 \leq s$, which is in contraction with our assumption that $S$ does not have any minimum. By the induction principle we conclude that $T = \mathbb{N}$. Since $S$ is nonempty, there is an $s \in S$. Moreover, $T = \mathbb{N}$ and we have $s \in T$, which leads us to $S \subset \mathbb{N} - \{1, 2, \dots, s \}$ which implies that $s \notin S$. This is a contradiction so $S$ must have a minimum.
\end{prf}

\end{document}